\chapter{Technical design}
\label{chap:architectural-design}
% repeat technical specifics of the architecture - choice of languages, frameworks, libraries and other tools
\section{Programming Language and Frameworks}
\label{sec:programming-language-frameworks}
% introduction to the importance of choosing programming lang + framework
% stating primary choices: TS, React, Koa, PSQL
\subsection{Programming Language}
\label{subsec:programming-language}
% why TS, compare maybe with python
\subsection{Frontend Framework}
\label{subsec:frontend-framework}
% discussion about choosing React (component-based arch, wide community support)
% comparison with alternatives
\subsection{Backend Framework and Server Environment}
\label{subsec:backend-framework-server-env}
% discussion about choosing KoaJS, compare with other backend framework (Express.js, Django/Flask) 
\subsection{Database Management System}
\label{subsec:dbms}
% discussion about choosing PSQL (robustness, complex queries, multimodel)
% contarst with other DBMS (MySQL, MongoDB, ..)
\subsection{Integration and Compatibility}
\label{subsec:integration-compactibility}
% discussion about integration of technologies together
% benefits of using unified tech stack
\section{Multi-tenancy and its possible approaches}
\label{sec:different-approaches-for-multitanency}
% discussion about multi-tenancy architectural approaches and what is multi-tanency in general.
\subsection{Multi-tenancy}
% general overview of the term with some examples

\subsection{Approaches}
% discuss possible approaches to multi-tenancy, detail each aproach's technical implementation - advantages and disadvantages

\subsubsection{Single database, single schema}
%Where all tenants share the same database and schema, but data is logically separated using tenant IDs in tables.

\subsubsection{Single database, multiple schemas}
%Each tenant has its own schema within a shared database, providing better data isolation while still sharing the same database instance.

\subsubsection{Multiple databases}
% Each tenant has its own database, maximizing data isolation and allowing for customizations but increasing complexity and potential resource usage.


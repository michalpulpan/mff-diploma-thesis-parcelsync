\chapter{Analysis}
\label{chap:analysis}

% brief intorduction of the purpose of chapter

\section{Requirements}
\label{sec:requirements}
% intro, requirements (ref https://engineering.futureuniversity.com/BOOKS%20FOR%20IT/Software-Engineering-9th-Edition-by-Ian-Sommerville.pdf) with descriptiion of what's functional/nonfunctional requirement 
\subsection{Functional Requirements}
\label{subsec:functional-requirements}
% functions of the software itself and behaviour in given situations

\subsection{Nonfunctional Requirements}
\label{subsec:nonfunctional-requirements}
% quality attributes of the software
% Multi-tenant architecture
% Docs
% CI/CD
% Source code (formatting, conventions)
% support for multiple carriers

\section{Programming Language and Frameworks}
\label{sec:programming-language-frameworks}
% introduction to the importance of choosing programming lang + framework
% stating primary choices: TS, React, Koa, PSQL
\subsection{Programming Language}
\label{subsec:programming-language}
% why TS, compare maybe with python
\subsection{Frontend Framework}
\label{subsec:frontend-framework}
% discussion about choosing React (component-based arch, wide community support)
% comparison with alternatives
\subsection{Backend Framework and Server Environment}
\label{subsec:backend-framework-server-env}
% discussion about choosing KoaJS, compare with other backend framework (Express.js, Django/Flask) 
\subsection{Database Management System}
\label{subsec:dbms}
% discussion about choosing PSQL (robustness, complex queries, multimodel)
% contarst with other DBMS (MySQL, MongoDB, ..)
\subsection{Integration and Compatibility}
\label{subsec:integration-compactibility}
% discussion about integration of technologies together
% benefits of using unified tech stack
\section{Different approaches for multi-tenancy}
\label{sec:different-approaches-for-multitanency}
% discussion about multi-tenancy architectural approaches


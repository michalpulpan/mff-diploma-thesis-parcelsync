\chapter{Deployment}
\label{chap:deployment}
% brief introduction into the chapter
In the fast-changing world of web applications, deployment plays a crucial role in the success of any software.
When building a \ac{SaaS} platform meant for various users in various environments, it is important to ensure that the platform is not only adaptable and scalable, but also robust and secure.
This necessity is the foundation on which this chapter is built.
Combining modern cloud technologies with good practices, this section explores how these elements are used in order to ensure smooth and efficient deployment process.
With a focus on \ac{IaC}, this chapter highlights how this method is used within the \ac{AWS} ecosystem providing in-depth look at the deployment procedure for both frontend and backend components discovering how they directly impact scalability, reliability, and security of the platform.

\section{Current Deployment Strategy}
\label{sec:current-deployment-strategy}
%  introduce the current deployment strategy of project
%  the deployment is managed using Infrastructure as Code (IaC) on AWS, use of specific services (Amazon S3 for hosting the static frontend and AWS Lambda for the backend functionality)
% how these components are organized into multiple stacks using AWS cloudformation

The deployment strategy for this projects leverages \ac{AWS} services with focus on \ac{IaC} to automate and manage cloud infrastructure.
Thanks to this approach, creating a consistent deployment process is achieved while reducing possibilities of human error and ensuring replication across different stages or environments.

Specifically, the strategy uses Amazon \ac{S3} for hosting the static frontend(s), \ac{AWS} Lambda for backend functionalities including scheduled background tasks and organising these diverse components into multiple stacks using \ac{AWS} CloudFormation.
Furthermore, this project improves its architecture with the inclusion of the PostgreSQL database using \ac{RDS}, Amazon Route 53 for domain routing, Amazon Certificate Manager for SSL/TLS certificate management and Amazon \ac{SES} for securing a high email delivery rate.
This structure allows for a well-controlled infrastructure, allowing quick adjustments if needed.

\subsection{Amazon Simple Storage Service for Static Frontend Hosting}
\label{subsec:amazon-s3-static-frontend}
% how Amazon S3 is used for storing and serving the static ReactJS app
The deployment strategy for the static frontend applications of the platform employs the Amazon \ac{S3} to host the ReactJS applications. Amazon \ac{S3} provides a reliable, scalable and secure solution for serving static content, making it ideal choice for hosting a \ac{SPA} applications like ours.
Both frontend applications (tracking page and dashboard) use very similar deployment strategies.
The \ac{S3} buckets are configured to serve a website with \texttt{index.html} with allowing public access and establishing removal policies to ensure that the buckets are destroyed when needed. 
However, we cannot provide access to the S3 bucket just like that.
Defining Amazon CloudFront distribution (Amazon \ac{CDN}) is vital for catching errors and unauthorised accesses with enforcing an SSL certificate managed by \ac{AWS} Certificate Manager.
Finally, DNS routing for applications is configured with Amazon Route 53, creating an A record pointing to the CloudFront distribution.

\subsection{AWS Lambda for Backend Services}
\label{subsec:aws-lambda-backend}
% use of AWS Lambda for the backend
% discussing how serverless architecture benefits the application, including scalability, cost-efficiency, and reduced operational overhead.

\subsection{AWS CloudFormation for Infrastructure Management}
\label{subsec:aws-cloudformation-infrastructure}
% the role of AWS cloudformation in managing the infrastructure.
% consistent, repeatable deployment processes

\section{Alternative Deployment Methods}
\label{sec:alternative-deployment-methods}
% This section should explore alternative deployment methods that could be used for the project. 

\subsection{Containerization}
\label{subsec:containerization}
% concept of containerization, particularly using Docker
% how containerization could be applied to both the frontend and backend

\subsection{Other Cloud Providers and Services}
\label{subsec:other-cloud-providers}
% Briefly touching on alternative cloud providers (like Azure or Google Cloud Platform) and their services that could be used for a similar deployment strategy.

\section{Continuous Integration and Continuous Deployment (CI/CD)}
\label{sec:cicd}
% Dedicate this section to the CI/CD processes implemented in the project, emphasizing the use of GitHub Actions.

\subsection{GitHub Actions for CI/CD}
\label{subsec:github-actions-cicd}
% how GitHub Actions are used for CI/CD
% details on how code changes in the repository trigger automated workflows for building, and deploying both the frontend and backend components

\subsection{Workflow Automation and Pipeline}
\label{subsec:workflow-automation-pipeline}
% specific steps involved in  CICD pipeline, including code commits, automated testing, build processes, and deployment to AWS services.

\chapter{Integrating SAP Business One}
\label{chap:integrating-sap-b1}

% problem description -> for the proof of context and real life texting was neccessary to build direct connection with SAP B1
% Communication with SAP isn't straight forward and is forbidden to write into SAP tables
% the decision was that the external software will be in charge of the synchronizations (call Parcelsync directly when needed, no webhooks or calls from the Parcelsync)

\section{Existing solutions}
\label{sec:existing-solutions}
% describe state of the art
\subsection{SAP Business One DI API}
\label{subsec:sap-b1-di-api}
% low level approach with direct access to the object
% was implemented in the bachaleor's thesis however did not fullfill the expectations and was pretty much phased out in favor of the SAP Business One Service Layer
\subsection{SAP Business One Service Layer}
\label{subsec:sap-b1-service-layer}
% Year after finishing the bachaleor's thesis, company moved to the newer version of SAP Business One (v9) which introduced SAP Business One Service Layer which finally brought simple REST-based approach for communicating with SAP with all mappings onto SAP objects and tables
\section{SAP Business One Service Layer Proxy with Database connector}
\label{sec:sap-b1-service-layer-proxy}
% description of the SAP Business One Server Layer Proxy which was created so that company can simply integrate Parcelsync. 
\subsection{Analysis}
\label{subsec:analysis}
% analysis of the proxy
\subsubsection{Functional requirements}
\label{subsubs:functional-requirements}
% func requirements of the proxy -> 
% - simpler authentication with basic role based access
% - proxy requests to the Service Layer
% - create database GETTER (for MS SQL DB with ODBC connector)
\subsubsection{Nonfunctional requirements}
\label{subsubsec:nonfunctional-requirements}
% deployment locally to the SAP B1 instance
% CI/CD
\subsection{Architecture}
\label{subsec:architecture}
% high level architecture of the proxy software
\subsection{Implementation}
\label{subsec:implementation}
% KoaJS app simmilar to the Parcelsync backend
\subsection{Deployment}
\label{subsec:deployment}
% Deployed as an containarized application into Docker
% Had to be local to the SAP B1 instance -> Linux VM on the server of company 


\chapter{Related work}
\label{chap:related-work}
% brief intorduction of the purpose of this chapter
% importance of understanding these works in the context of this project
This chapter dives into the overview of existing solutions within the scope of parcel logistics in the dispatch process, generating labels, and shipment tracking, with a focus on the Czech market.
Understanding these projects and their limitations helps us to understand the market gap that this platform is filling.
Particularly in offering a cloud-based multi-tenant solution integrating customised tracking page and email notifications for recipients.
The overview underscores the importance of innovating beyond current offerings, which primarily lack features such as a simple dashboard for data viewing, custom tracking pages for improved customer communication, and automated email notifications.

\section{Related projects}
\label{sec:related-projects}
% boundaries of Czech environment and usage of local carriers - PPL, Packeta, Česká Pošta
% None of those implementing custom tracking page for customer communication 
% None of those implement auto e-mail sender to customer
% None of these is in cloud serving working as a multi tenant solution that customer doesn't have to care about

Projects handling data communication with carriers are, of course, heavily biased with the demographics they are targeting. 
Although the process is usually very similar, shipping companies and customers with their e-Commerce platforms or \ac{ERP} solutions are different.
Hence, we will limit ourselves to the Czech logistics environment, where several systems facilitate the integration with local carriers. However, these solutions often fall short in several areas:
\begin{itemize}
    \item None operates as a cloud-based multi-tenant solution that offers the business a hassle-free platform for their logistic needs without the necessity for in-house infrastructure or maintenance.
    \item Usually, they require a distinct approach and data model for different carriers. This makes it more difficult to integrate.
    \item They typically do not provide a custom tracking page for end-to-end customer communication, missing an opportunity to enhance the customer experience with a branded informative user interface.
    \item Current solutions typically do not allow a single company to use multiple shipping carrier contracts through set of different API credentials. This limitation can be problematic for businesses with multiple warehouses, each requiring different shipping carriers due to their unique logistics needs.
\end{itemize}

Let us take a look at some options available in the Czech market.

\subsection{Balíkobot.cz}
\label{subsec:balikobot}
% integration directly into the ERP/ecommerce system
% no possibility to edit parcels outside of the ERP
% offloading processes from ERP
% SAP is fairly small and hardly accessible from the "outside" so it's not a great option
% usually doesn't directly implement data structure of the ERP
Offers integration directly into \ac{ERP}/e-Commerce systems, however lacking the flexibility to modify parcels outside of these systems.
Requiring that the user using this software can modify the source data in \ac{ERP} and not being able to use any Balíkobot user-interface might be a strong limitation.
The next limitation might be the label format provided by Balíkobot.cz.
This solution generates its own label layout, validated by the carrier, of course, but sometimes a different layout might lead to inefficiency or even errors at the sorting centres or when loaded to the delivery vehicle.
On the other hand, the amount of integrated carriers and \ac{ERP} integrations makes Balikobot very easy to start using.
One thing to consider is that Balíkobot does not handle customer communication at any level.
It is only strictly used for data transfer and printing the shipping labels.

Overall, Balíkobot is definitely a considerable solution, but being integrated directly into the \ac{ERP} makes it a bit slow to use and is usually inaccessible from outside of the company network when needed.
And, the most importantnly, it deals only with company processes, not with customer communication and presentation.

\subsection{LabelPrinter.cz}
\label{subsec:labelprinter}
% on premise windows service run at client
LabelPrinter, as the name suggests, is designed for label printing and data transfer. Like Balíkobot, LabelPrinter functions solely as warehouse software to transfer data from companies to carriers.
Additionally, it operates only as an on-premise Windows service runtime at the client's computer.
This approach requires local infrastructure and maintenance, potentially increasing operational overhead in small to medium-sized enterprises, limiting scalability and accessibility.
Additionally 

% Zminit inhouse solution nebo to nechat byt??

It might also be beneficial to mention that companies usually also use "in-house" solutions, which are generally a bespoke set of scripts designed for data transfer without additional features like customer communication. 
These are often difficult to maintain and lack functionalities such as shipping status mapping, essential for recognising final delivery statuses.


\section{Addressing the shortcomings of existing solutions}
\label{sec:addressing-shortcoming-existing-solutions}
% - Abstract data model allowing for unified data format sent to the software
%- Dashboard with data overview seeing all the errors and direct label/consignment list printing accessible from outside network not beeing hosted locally
%- Unification of parcel statuses into few statuses only so that the user doesn't have differentiate between different carriers and translate statuses for their purpose
%- Branded parcel tracking page
%- Branded e-mail tracking notification
%- Not having to care about the deployment
%- Role based access for users
%- Having multiple projects with different carrier API credentials and settings

Problems of data communication with carriers present numerous challenges with existing solutions, particularly in their ability to scale and offer a seamless user experience across different carriers.
This platform confronts these issues, presenting a new approach to the landscape of automation logistic expedition and post-purchase experience.

\subsection{Unified data model}
Existing solutions often lack a unified approach to data handling, which complicates integration with different carriers. 
Our platform introduces a unified data model normalising data formats across all carriers, simplifying the integration, leaving the complexity of understanding different models to the project itself.
This fits very well with the goals presented in the \hyperref[subsec:project-goals]{Project goals} section.
Specifically, \textbf{G1} and \textbf{G4} are closely related to the complex integration of external systems.

\subsection{Centralized dashboard}
Having of a user-friendly dashboard makes it convenient to monitor and manage shipments. However, in the presented solutions, most of it is left to the existing \ac{ERP} which is usually not made to handle logistics data.
This shortcoming again reflects very well few of our \hyperref[subsec:project-goals]{goals}, \textbf{G1} and \textbf{G2}. A cloud based multi-tenant solution trying to streamline logistics operations for the operators in the warehouse lacking a user-friendly dashboard would be very difficult to operate and would probably require each integrator to create their own interface for data presentation, which we do not want.
Providing a modern web application with a dashboard accessible from anywhere gives a great competitive advantage and improves the platform operator's experience. %user experience. 
With an overview of all data sent and retrieved from carriers and direct functionalities for label and consignment list printing, the user does not have to use any other interface when working with parcels.

\subsection{Parcel status unification}
The set of parcel statuses provided by the carriers is, of course, very distinct. Every carrier API is different; hence, it provides different data and communicates in a different way.
Consolidation of various statuses into a standardised set, allowing users to easily understand and manage shipments without getting lost in carrier-specific environments.
Supporting our goal of \hyperref[subsec:project-goals]{\textbf{G4}} simple integration with an existing system, where a company typically tries to convert shipment statuses from a carrier into a more uniform format. It will, of course, also help with the practical demonstration of the integration resulting from \hyperref[subsec:project-goals]{\textbf{G5}}.

\subsection{Branded tracking and notifications}
Enhancing post-purchase communication is often overlooked, yet it directly supports the objective outlined in \hyperref[subsec:project-goals]{\textbf{G3}}.
Businesses usually leave this channel to the third party, such as carriers, and focus on prepurchase marketing, which is usually, in digital marketing, standard \ac{PPC} adverts. 
However, \ac{PPC} campaigns often rely on cookies to target and retarget ads based on user behaviour.
As this segment becomes more regulated and with the reopening of the discussion on updating ePrivacy legislation \footnote{The EU's ePrivacy Regulation, initially established in 2002, along with the General Data Protection Regulation, both influence the tracking of visitors from the EU.} in the European Union, obligations start to come from ad service providers themselves, such as the Google V2 consent mode \footnote{The Google V2 consent mode allows website owners to adjust how their Google tracking tags behave based on the consent status of their users. The consent status might be set through the cookie bar.}. 
Leaving aside the fact that most smaller online retailers generally do not even reflect the changes and ignore them, thus putting themselves at a disadvantage in online advertising, a large proportion of customers are becoming more and more difficult to reach.
%However, as this segment becomes more regulated, some customers are difficult to reach.
Communicating with a customer, when the most important part of the purchase is happening, could be a key to greater brand recall in today's advertising overload. 

\subsection{Simplified integration}
The technical challenges and costs of implementing and installing a software solution on premise might act as a barrier to many businesses. 
The cloud-based solution eliminates these obstructions while supporting both the objectives outlined in \hyperref[subsec:project-goals]{\textbf{G2}} and \hyperref[subsec:project-goals]{\textbf{G4}}.

\subsection{Role-based access control}
Security across the platform demands a control over user permissions.
By implementing role-based access, we can improve security and ensure that users have the appropriate permissions for their role.

\subsection{Versatile carrier communication}
Businesses often work with multiple shipping carriers with different contracts and settings. 
For example, each warehouse might require a different contract with the carrier when it is in a different region.
This can be difficult to manage. 
Hence, it is necessary to implement a solution that can manage multiple locations (projects) in one profile with different carrier API credentials and carrier settings while distinguishing between shipments and users from different locations.
This can simplify integration complexity in a multi-location environment while supporting the \hyperref[subsec:project-goals]{\textbf{G4}} goal.

% 


\chapter{Implementation}
\label{chap:implementation}

\section{Project Structure}
\label{sec:project-structure}

\subsection{Monorepo and it's advantages}
\label{subsec:monorepo-advantages}
% Detail the specific advantages of using a monorepo for this project, such as simplified dependency management, streamlined workflows, and easier code sharing between different parts of the project.

\section{API Service (Backend) Implementation}
\label{sec:api-service-implementation}
% Dedicate this section to the API service part of the project, which is built using KoaJS.

\subsection{KoaJS Framework}
\label{subsec:koajs-framework}
% brief discussion the koajs and why it was chosen for the API service. Include details on its features, benefits, and how it supports the project requirements, such as handling HTTP requests and middleware management.
% reference to the analysis

% ...

\subsection{API Design and Endpoints}
\label{subsec:api-design-endpoints}
% Discuss the design of the API, including the structure of endpoints, RESTful principles applied, and any particular design patterns or practices used.

\section{Web Client Implementation}
\label{sec:web-client-implementation}
% This section should focus on the implementation of the web client, developed using ReactJS.

\subsection{ReactJS Framework}
\label{subsec:reactjs-framework}
% Highlight its features, such as component-based architecture, and how it facilitates the development of interactive user interfaces.
% reference to the analysis

\subsection{Client-Side Routing and State Management}
\label{subsec:client-side-routing-state}
% Elaborate on the implementation of client-side routing, state management, and any other significant aspects of the web client application.

\section{Infrastructure as Code}
\label{sec:infrastructure-as-code}
%  Infrastructure as Code aspect of the project, detailing how it is integrated and managed within the monorepo.

\subsection{IaC Tools and Configuration}
\label{subsec:iac-tools-configuration}
% the tools and technologies used for IaC (Serverless), and how they are configured and utilized within the project.


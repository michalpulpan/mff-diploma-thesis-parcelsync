E–commerce trh v oblasti prodeje zboží zaznamenal růst, který výz\-namně ovlivnil celý maloobchod.
Prodejci se dostali do situace, kdy nestačí produkt pouze nabízet, nýbrž je nutné budovat značku a posilovat vztahy se zákazníkem.
Značky se tak musí odlišit od konkurence, upoutat na sebe pozornost a zároveň klást důraz na udržení kvality péče o zákzníka.
%Po velmi dlouhou dobu byla logistika vnímána prodejci pouze jako interní proces, za který nese zodpovědnost jen do chvíle předání zásilky přepravci.
Logistická komunikace během doručování objednávky je zpravidla přenechávána přepravci, ale představuje důle\-žitý prvkem v životním cyklu objednávky.
Nabízí možnost zvýšit povědomí o  značce a posílit vazbu se zákazníkem. 
Tato práce navrhuje řešení spočívající v automatizaci procesů datové komunikace s přepravcem a v jejím využití jako potenciálního marketingového kanálu.
Představuje SaaS platformu testovanou v reálném firemním prostředí na tisících zásilkách.
Ta umožňuje prodejcům spravovat přenos dat k přepravcům, tisknout přepravní štítky a efektivně sledovat stav zásilek.
To vše s upravitelnými notifikačními e-maily a sledovací stránkou určenou příjemci.
Tento přístup nejenže zefektivňuje zaběhnuté procesy expedice, ale zároveň posiluje značku během ponákupního marketingu v průběhu přepravy objednávky.
